

\documentclass{article}
\usepackage{dsfont}
\usepackage{bm}

\title{Structor Theory}
\author{Shiva}
\date{\today}

\newtheorem{definition}{Definition}
\newtheorem{proposition}{Proposition}

\begin{document}
\maketitle

In arithmetic we manipulate numbers using addition and multiplication. However, it is still a deep mystery how these two operations are fundamentally related. In particular, we do not know how the multiplicative structure of numbers arises as a result of addition. 
 
In Nature we observe discrete numbers increasing, not by additive or multiplicative processes, but by an underlying continuous evolution of structure. For example, the number of leaves on a tree is determined by its branching structure. 

In this paper we define a class of elements that combine both magnitude and structure. By studying their properties we hope to uncover a more fundamental context in which numbers themselves may be embedded, and through which we might gain a new perspective into their arithmetic properties. 
 
\begin{definition}
A {\bf structor} is an ordered pair $(m,r)$, where $m \in \mathds{R}$ and $r \in \mathds{R}^{+}$ 
\end{definition}

A generalization of a vector, a structor represents {\bf structural evolution} and {\bf magnitude} by encoding a rooted tree with weighted nodes. Structural evolution is parameterized by $r$, and is the rate of change of a node's weight with respect to its parent. Magnitude is given by $m$, and is the sum of all the node weights. 

Starting with a root node of zero weight, continuously evolve the weights of all nodes, according to the rate $r$, until the tree has a total weight of $m$.

Nodes are created and removed according to the following rule: Whenever any node weight crosses an integer multiple, spawn from it a new child node of zero weight, and whenever any leaf node weight reaches zero, remove it.

\begin{proposition}
The set of all structors $\mathds{S}$ is a {\bf group}, where $(m_{1}, r_{1}) \circ (m_{2}, r_{2})$ is defined by taking structor $(m_{1}, r_{1})$ as the starting point and evolving it by $(m_{2}, r_{2})$ 
\end{proposition}

Question: Does  $\mathds{S}$ have any non-trivial finite subgroups?
Ans: These would have to be generated by pairs of structors $(m_{1}, r_{1}) \circ (m_{2}, r_{2})$, where $m_{1} = -m_{2}$, which we call zero-structors

\end{document}